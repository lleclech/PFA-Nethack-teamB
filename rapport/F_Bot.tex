\chapter{Bot}

Les bots sont completement indépendants du jeu. Cela signife qu'ils peuvent être écrits dans n'importe quel langage de programmation, les seuls prérequis étant de pouvoir ouvrir une socket et d'envoyer des chaînes de caractères.

\section{Développer un bot}

\subsection{Etablir la communication avec Nethack}
Le bot et Nethack communiquent via une socket de type internet dont le port est 4242. Ce port est écrit en dur dans le code.

\subsection{Envoyer des ordres}
Le bot doit utiliser un langage haut niveau pour communiquer ses ordres au jeu.

\begin{description}
\item[SEARCH :] permet de rechercher une porte secrète.
\item[OPEN :] permet d'ouvrir une porte.
\item[FORCE :] permet de forcer l'ouverture d'une porte.
\item[MOVE :] permet de se déplacer.
\end{description}

Certains ordres nécessitent d'être suivi d'une direction pour être exécutés. Ces directions ont également été traduites, elle correspondent aux points cardinaux en anglais et en majuscules (NORTH, SOUTH\_WEST etc). Nous avons également ajouté UP et DOWN, qui correspondent aux escaliers du donjon et permettent d'accéder aux niveaux supérieurs et inférieurs.

Le bot ne peut envoyer qu'un ordre par tour. Une fois celui-ci envoyé il faut donc se mettre a l'écoute du jeu qui va nous envoyer un certain nombre d'information (par exemple la carte du jeu). La liste des ordres peut-être modifiée via le Bot Handler.

\subsection{Information renvoyée par le jeu}
Lorsque le jeu reçoit un ordre, il l'applique, puis renvoie la carte mise à jour sous la forme d'une chaîne de caractère. Si l'on souhaite avoir un bot un peu évolué, il peut être interessant de la stocker. 

la chaine de caractère envoyée par le jeu a pour forme:

\begin{verbatim}
START
DUNGEON_LEVEL 1
MAP_HEIGHT 6
MAP_WIDTH 10
START MAP
          
   ----   
  |...@+  
  |....|  
   ----   
          
END MAP
END
\end{verbatim}



\section{Exemple de bots en java}

Nous avons codés deux bots relativement simple en java. Ceux-ci possèdent un tronc commun important. 
Le rôle d'un bot est de se connecter au jeu sur le port adéquat, et de lui envoyer son nom pour ammorcer le dialogue. Le jeu modifié lui répond alors par une carte du type décrit précédemment. Le bot est donc une boucle principale de réception de cette chaîne, d'analyse et de production de sortie. Nous avons dupliqué une grande partie de ce code, puisque la connection, l'analyse syntaxique de la carte, la production et l'envoi des messages sont des foctionnement universels aux bots pouvant jouer au jeu. Par ailleurs, les bots sont équipés d'un système de log qui peut être activé ou non afin d'en débogguer la sortie. 

Nous distinguerons ensuite les deux algorithmes différents décrivant la logique de sélection du coup à jouer.

\subsection{Diagramme de classe}

Voir l'annexe.
 
\subsection{Description}

Voici une description rapide de la plupart des classes du bot:

\begin{description}
\item[Map :] Cette classe permet de stocker et de manipuler la carte.
\item[Bot :] Cette classe permet de programmer le comportement du bot.
\item[InputOutputUnit :] Cette classe permet de gérer les connexions avec le Bot Handler.
\item[SquareType :] Cette classe contient tout les types de case pouvant être présent sur la carte.
\item[Protocole :] Cette classe contient les commandes à envoyer au Bot Handler.
\item[StepMap :] Contient certaines methodes permettant d'ameliorer l'algorithme du bot.
\item[Direction :] Cette classe contient des variables de directions.
\item[Variables :] Cette classe contient des variables diverses (comme ACTION\_RESULT par exemple).
\end{description} 

\subsection{Algorithme}

Les deux bots sont des bots stochastiques. Tous deux tirent un dé pour déterminer si ils font ou non une recherche, sinon ils se déplacent. La différence d'algorithme réside dans la sélection de la direction du mouvement. Le premier bot choisi une direction au hasard parmis toutes celles disponibles, et la joue. Il ne maintient aucune donnée persistante sur les coups passés et se contente de la carte actuelle. Le second algorithme procède de la même façon, mais compte les passages qu'il effectue sur chaque case, et sélectionne uniquement parmis les cases où il est le moins passé. De cette façon, il évite de nombreux allers-retours ; les couloirs, par exemple, sont systématiquement parcourus sans perdre de mouvement, et les nouvelles issues sont plus rapidement explorées, ce qui crée une différence importante de performance entre les deux stratégies. 

\subsection{}
