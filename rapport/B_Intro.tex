\chapter*{Introduction}
\addcontentsline{toc}{chapter}{Introduction}

Nethack est un jeu de type rogue-like sorti en 1987. Il est le résultat d'un travail collaboratif via internet, d'où il tire son nom. Le jeu consiste à incarner un héro chargé de retrouver l'amulette de Yendor, qui se trouve au 26ème niveau d'un donjon aléatoirement généré, rempli de pièges et de monstres. Le jeu se joue via un terminal texte, chaque caractère représentant une case du jeu. Le but du projet était de proposer une plateforme de simulation de robots, dans le but de pouvoir tester et comparer les performance de différentes solutions face à une forme simplifiée du jeu de départ. 

Il a donc fallu concevoir un certain nombre d'outils pour d'une part faire en sorte que le jeu, initalement conçu pour l'interface clavier, intéragisse avec des bots, puis conçevoir les bots en question, et enfin permettre de collecter et exploiter les informations de chaque partie, de les restituer de telle sorte que l'on puisse comparer les performances des bots.



