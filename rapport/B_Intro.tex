\chapter{Introduction}

Étant redoublant en première année d'école d'ingénieur à l'ENSEIRB-MATMECA, j'ai eu la chance de pouvoir faire un stage de 6 mois à Logica à la place de mon second semestre de première année. Ce choix de travailler chez Logica s'est fait assez naturellement, car je voulais découvrir l'univers d'une grande entreprise, et principalement une SSII. J'ai intégré l'équipe de tests et recette de la Banque postale chez Logica, la Testing Factory. Durant ce stage, j'ai fait de la conception et de l'exécution de tests, et du développement sur des automates de tests. J'ai également participé au développement d'une application pour la Banque Postale. Enfin j'ai travaillé occasionnellement sur des mises à jour d'applications internes à Logica.

La phase de test est souvent négligée lors du développement d'un projet, Bien qu'elle soit indispensable, elle est  souvent longue à effectuée et donc coûteuse. La Testing Factory possède des équipes spécialisées dans la conception et l'exécution de test, cette offre de service permet donc de répondre aux besoins du client avec une bonne réactivité..

La première partie de ce rapport décrit la société Logica et son positionnement sur le marché très concurrentiel des SSII. La deuxième partie regroupe les tâches qui m'ont été confiées ainsi qu'une description de la méthodologie TestFrame, véritable épine dorsale de la Testing Factory.


