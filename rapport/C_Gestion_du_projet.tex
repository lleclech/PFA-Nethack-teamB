\chapter{Gestion du projet}

\section{outils de gestion utilisés}

Gérer un groupe de projet de sept personnes est évidemment assez complexe, nous avons donc utilisé plusieurs outils pour nous aider. Après la séparation en deux groupes, certains outils ont été abandonnés car ils n'était plus très interessant.

\subsection{Gestionnaire de version}

Nous avons utilisé le gestionnaire de version git pour notre projet. Le gestionnaire permettant la création de branches et leur fusion de façon simple et automatique, nous avons eu l'idée d'utiliser git pour organiser le projet par branche. Une branche de base stable devait servir de version de référence, à partir de laquelle chacun crée sa propre branche de développement. Les branches devaient être fusionnées une fois les développement principaux terminés. Cette organisation permettait d'individualiser les développements ; elle a le mérite d'être sûre, mais elle s'est révelée peu efficace : les branches mettant trop de temps à être fusionnées, la branche de départ est restée longtemps vide, ralentissant la circulation d'information et la mise au point entre les modules dépendants. Par la suite et renforcé par la réduction d'effectif, nous avons rassemblé tous les développements dans une unique branche de développement, fonctionnement simple mais qui s'est révélé suffisant pour nos besoins. 

\subsection{Gestion des ressources}

Lors de la première partie du projet, nous avons utilisé le logiciel planner pour faire un diagramme de gantt. Ce logiciel est assez interessant car il permet de mettre a jour le diagramme de manière très simple et il affiche les différentes informations (comme par exemple l'avancement d'une tache) de manière claire.
Cependant, nous avons un peu délaissé cet outils dans la deuxième partie du projet car, à quatre, il était aussi simple de se répartir les taches oralement. L'ajustement de répartition par rapport à l'avancement réel en a été facilité.

\section{Mode de développement}


