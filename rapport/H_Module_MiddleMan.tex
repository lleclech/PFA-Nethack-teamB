\section{Middle Man}
\paragraph{}
Le middle man est un intermédiaire entre le noyau du jeu nethack
et les sources permettant le dialogue avec les bots. Il s'aagit d'un module C,
couplé d'un patch qui permet de l'installer dans le jeu.

\paragraph{}
Dans le code de nethack,
il est installé comme une substitution aux événements qui sont relatifs
à l'interface utilisateur. Les fonctions du middle man sont appelé lorsque le
noyau attend une entrée clavier ou affiche un caractère sur le terminal texte.
De cette façon, il est impossible pour le bot de "tricher" : les informations 
collectées correspondent aux informations qu'un utilisateur peut lire à l'écran.

\paragraph{}
Lorsqu'il intercepte une attente d'entrée clavier, le middle man active le
bot handler en lui fournissant la carte qu'il a créé au fur et à mesure des affichages.
Il attend de la part du bot qu'il effectue des actions. 

\paragraph{}
Le middle man contient également d'autres points d'entrée permettant les collectes 
statistiques. Ces informations correspondent aux portes cachées dans le niveau, aux portes
ouvertes pendant la partie, aux portes secrètes totales, et à la plus grande
profondeur du donjon atteinte.

\paragraph{}
Un nombre de tour maximal est fixé par une constante du module.
Après ce maximum, la partie est interrompue, et les statistiques sont transmises aux modules de
base de données.

