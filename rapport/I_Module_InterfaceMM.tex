\section{Interface Middle Man} 

Ce module s'occupe recuperer les informations interessantes d'une partie, les formates, et les stocke via le DB manager dans la base de donnée.
Il est principalement composé d'une structure "AllData"permettant de ranger les informations d'une partie, et d'un ensemble de fonction
qui recuperant les informations et les places dans "AllData".

\subsection{AllData}

Voici un exemple de structure AllData:

\begin{verbatim}
struct AllData
{
  int id_party;
  struct Mod_Info mod;       // Informations sur le mod utilisé
  struct Bot_Info bot;       // Informations sur le bot utilisé
  struct Date date;
  int door_lvl;         // Nombre total de portes dans le niveau généré
  int door_disc;    // Nombre de portes découvertes
  int steps;              // Nombre de pas effectués
};
\end{verbatim}
Cette structure contient toutes les informations que l'on souhaite sauvegarder dans la base de données. Il est à noter que cette structure correspond à un environnement de jeu et que si on décide de changer les informations a ranger dans la base de données, il faut modifier cette structure.
 
\subsection{Fonctions de récuperation}

voici les fonctions de récuperation pour la structure AllData pris en exemple: 

\begin{description}
\item[assign\_nb\_door\_level :] Ecris le nombre de portes découvrables pendant la partie dans la structure de données

\item[assign\_nb\_door\_discovered :] Ecris le nombre de portes découvertes pendant la partie dans la structure de données. 

\item[assign\_nb\_steps :] Ecris le nombre de pas effectués pendant la partie dans la structure de données.

\item[assign\_nb\_monsters\_generated :] Ecris le nombre de monstres générés pendant la partie dans la structure de données.

\item[assign\_nb\_monsters\_killed :] Ecris le nombre de monstres tués pendant la partie dans la structure de données.
\end{description}
