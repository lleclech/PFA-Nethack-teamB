\chapter{Logica 4ème SSII européenne}

\section{Qu'est-ce qu'une SSII?}
Une Société de Service en Ingénierie Informatique est une société spécialisée dans les prestations de service dans le domaine informatique. Dans les années 1970, le développement de ces prestataires exerçant des activités d’étude et de conseil en matière de systèmes informatiques s’est intensifié. Elle peut englober plusieurs métiers (conseil, conception et réalisation d'outils, maintenance ou encore formation) et a pour objectif principal d'accompagner une société cliente dans la réalisation d'un projet.
Les SSII trouvent leurs structures clientes en répondant à des appels d’offres qu’elles remportent face aux entreprises concurrentes. Elles peuvent ainsi fournir des missions à leurs employés prestataires. D’un point de vue participation à un projet, elles peuvent être impliquées dès le départ, s'associer à sa réalisation en fournissant des équipes de développement, et en assurer le contrôle qualité.
Le prestataire ou consultant est quant à lui souvent spécialisé dans un domaine fonctionnel ou dans une technologie. La diversité de ses prestations lui assure sa montée en compétence et par conséquent une augmentation du prix de sa prestation.
Ces entreprises ont connu dans les années 1990 et 2000 une croissance démesurée, souvent à deux chiffres (sur l'année 2009-2010 Logica a vu son chiffre d'affaire croître de 7\% et ce malgré la crise).

Les offres des SSII sont structurées autour de huit prestations:
Etudes et Conseil – Progiciel – Intégration de Systèmes – Assistance Technique – Service et Réseaux à Valeur Ajoutée – Infogérance ou Facilities Management – Distribution, maintenance et évolution des matériels et des réseaux – Formation

Au sens plus large, les SSII tirent leurs chiffres d’affaires des prestations suivantes:
\begin{description}
\item[Etudes et Conseil :]pour accompagner les clients d’amont en aval, de la réflexion post-stratégique à la réalisation de leurs projets. Il peut leur être prodigué des conseils en organisation, en processus métier, en conduite du changement, en technique ou en recherche et développement externalisé.

\item[Intégration de systèmes :]où les domaines de compétence vont de l’intégration au développement des systèmes d’information, en passant par la conduite de projets et une expertise technique sur les sujets les plus pointus. Ils proposent de l’architecture des systèmes d’informations, du développement d’applications, de la mise en place de progiciel de gestion intégré permettant de gérer l’ensemble des processus opérationnels d’une entreprise,  vente de licences de logiciels, assistance technique, mise en place de solutions de communication entre divers systèmes informatiques hétérogènes.

\item[Infogérance ou Facilities Management :](externalisation ou outsourcing) qui consiste à confier ses prestations informatiques à un tiers. Cette prestation comprend des activités de tierce maintenance applicative (TMA) qui s’occupe de la maintenance et de l’évolution des applications, des activités de tierce recette applicative (TRA) et testing qui s’occupe en externe des tests et de la qualité logicielle, des activités de gestion des infrastructures comprenant la maintenance, l’hébergement et la gestion des réseaux et pour finir des activités de BPO (Business Process Outsourcing) consistant à externaliser les processus métier (ressources humaines, comptabilité …).

\item[Formation, assistance aux utilisateurs.]
\end{description}

\section{Et Logica?}

\subsection{Bref historique}
\begin{figure}[b]
  \begin{center}
    \includegraphics[width=12cm]{Images/fusion_logica.jpg}
  \end{center}
  \caption{La création de Logica}
  \label{la création de Logica}
\end{figure}
\clearpage
\begin{description}
\item[1983 :]      La société regroupe l’ensemble de ses filiales sous le nom d’Unilog.

\item[1987 :]     Unilog commence son expansion sur le territoire national et ouvre à Lyon sa première agence régionale. Création de l’activité Consulting afin de répondre à la volonté d’Unilog d’accroître sa capacité dans le conseil. Cette activité deviendra Logica Management Consulting.

\item[1988 :]      Entrée en bourse du second marché où la société réalisera en 1998 la meilleure performance boursière. En Juillet 1998, Unilog acquiert le groupe Integrata (800 personnes - top 20 en Allemagne). En 1999 elle accède au premier marché de la bourse de Paris.

\item[1996 :]      Les filiales régionales d’Alcatel TITN Answare et de Sinorg rejoignent la société. Au fil des années Unilog poursuit son développement par croissance externe avec Cesia en 1997 et NSA (société de 270 personnes) en 1999.

\item[1998 :]     Unilog s’installe en Allemagne, la société acquiert le groupe Integrata qui fait partie des 19 premières sociétés de conseil, d’ingénierie et de formation en Allemagne 

\item[2000 :]     Unilog adopte un nouveau logo et une nouvelle signature qui résume son positionnement : « À problèmes uniques, solutions uniques ». La création d'Unilog Université offre un cadre commun à la formation de l’ensemble de ses collaborateurs, depuis l’ingénieur débutant jusqu’au Top management. Unilog acquiert la société suisse GDI (100 personnes, 10 millions Euros de chiffre d'affaires), société de services informatiques implantée à Genève et à Lausanne. 
\item[2000 :]	  En janvier 2000, Unilog crée l'ESCAN (European Software Companies Allied Network), la première alliance européenne en conseil et ingénierie informatique, née de la coopération entre Engineering Ingegneria Informatica (Italie), Ibermàtica (Espagne) et Unilog. Cette alliance, qui pèse près de 650 millions Euros de chiffre d'affaires et rassemble 7 250 professionnels entend répondre aux nouvelles exigences du marché européen. 

\item[2000 :]     Le 6 avril 2000, Unilog acquiert la société allemande VSS : CA additionnel de 26 millions Euros et 250 collaborateurs. L'ensemble des entités allemandes de la Société représente désormais 1 150 collaborateurs et se place dans le top 10 du secteur. 

\item[2001 :]     Le 31 mai 2001, Unilog annonce l'acquisition de la division technology de la filiale anglaise de March First USA, spécialisée dans le conseil technologique et l'intégration de systèmes autour d'Internet. La nouvelle entité compte 150 collaborateurs. 

\item[2004 :]     Le 9 juillet 2004, Unilog et Avinci signent un protocole d'accord visant au rapprochement des deux entités et ainsi à la constitution en Allemagne d'une société de conseil et d'ingénierie, capable de rivaliser avec les meilleures. 

\item[2006 :]     Le 10 janvier 2006, LogicaCMG rachète Unilog. Ensemble, les deux sociétés donnent naissance à un leader européen des services informatiques 

\item[2006 :]     Le 1er mars 2006, Didier Herrmann prend officiellement ses fonctions de Président du Directoire d'Unilog SA, tandis que Gérard Philippot rejoint le Conseil de Surveillance en tant que Vice-Président

\item[2006 :]	  Le 16 octobre 2006, acquisition de la société suédoise VM-Data.

\item[2008 :]     Le 1er janvier 2008, Andy Green est nommé président du groupe LogicaCMG. Il succède à Martin Read. 

\item[2008 :]     Le 28 janvier 2008, LogicaCMG met en place la nouvelle division Services  d'externalisation afin de répondre à la demande de ses clients 

\item[2008 :] 	  Le 27 février 2008, Unilog devient Logica.  Le groupe LogicaCMG, qui opérait depuis le 10 janvier 2006 sous la marque Unilog en France, annonce l’adoption d’une nouvelle marque à l’échelle internationale : Logica. 

\item[2012 :]	  Logica est racheté par CGI, une société de service Canadienne.
 
\end{description}

\subsection{Positionnement et Activité}

Logica propose une très large palette d'activité, sur de nombreux secteurs: Banque, Distribution, Energie, Pharmacie, Secteur Public, Télécommunication... En Europe, le groupe est principalement présent en France, au Royaume-Unis, et aux Pays-Bas, et dans les pays scandinaves. Le groupe emploie plus de 40000 personnes dans 36 pays. En France, Logica a pour client la totalité du CAC 40, à l'exception de son concurrent Cap Gemini. C'est également un prestataire important de l'état puisque celui-ci représentait un tiers du chiffre d'affaires de 2009 de Logica France.  
\begin{figure}[h]
  \begin{center}
    \includegraphics[width=15cm]{Images/chiffre_d_affaire_2009.jpg}
  \end{center}
  \caption{La répartition du chiffre d'affaire de Logica en 2009}
  \label{La répartition du chiffre d'affaire de Logica en 2009}
\end{figure}
\subsection{Les métiers au sein de Logica}

\begin{center}
\begin{tabular}{|c|c|}
\hline
\begin{tabular}{c} \begin{large}\begin{bf}Consulting\end{bf}\end{large}\\ 3000 consultants\\Dont 1000 en France\\4ème acteur du conseil en france\\une organisation par type d'activité\\
\end{tabular}  
&
\begin{tabular}{c} \begin{large}\begin{bf}Outsourcing\end{bf}\end{large}\\ Un réseau de quatre \\centres de services en france\\plus de 1000 collaborateurs\\une croissance de 50\% par an \\
\end{tabular}
  
\\

\hline
\begin{tabular}{c} \begin{large}\begin{bf}Integration de Systèmes\end{bf}\end{large}\\ 6200 collaborateurs \\ Une organisation par secteur d'activités \\Une présence nationale : 20 implantations \\Des environnements fonctionnels et \\techniques variés dans chaque entité \\
\end{tabular}  
&
\begin{tabular}{c} \begin{large}\begin{bf}Infrastructure\end{bf}\end{large}\\ une entité dédiée\\plus de 1000 collaborateurs \\une mission: intégration, administration, \\et exploitation des systèmes \\ d'information des clients\\
\end{tabular}\\
\hline  
\end{tabular}
\end{center}


\subsection{Logica à Bordeaux}
\begin{wrapfigure}[6]{l}{4cm}
\includegraphics[width=4cm]{Images/Logica_Bordeaux.jpg}
\end{wrapfigure}

Il existe trois entités distinctes à Bordeaux. L'entité Centre Service de Gironde (CSG) fait partie des quatres centres de services nationaux qui sont donc affectés spécifiquement à la production.

L'entité Grand Angle (GA), une division de l'entité Services de Technologies de l'information, qui est affctée au développement d'un progiciel.

J'ai intégré à Logica la troisième entité, Sud-Ouest Languedoc Roussillon (SOL), Une "business division" du groupe, regroupant les agence de Bordeaux, Montpellier, Pau, et Toulouse. Cette division intervient pour des grands groupes des secteurs de l'industrie et du tertiaire, (comme par exemple la Banque Postale). Cette entité possède environ 500 employés.


