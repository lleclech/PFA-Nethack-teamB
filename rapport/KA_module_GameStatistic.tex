\section{Game Statistic}

Après chaque partie, les informations de la partie sont enregistrée. Toutes les données sauvegardée dans la base de données permettent de calculer les donées statistiques liées au bot et nous permettent de comparer leurs performances dans un environnement de règle donné..

Il existe plusieurs types de paramètre pour les statistiques:\\

\begin{description}
\item[Celui de bot]
  \begin{itemize}
  \item le type du bot, i.e., Algorithms/Strategies utilisées
  \item les parametres du bot, i.e. paramètres de l'algo/du strategie
  \item etc
  \end{itemize}
\item[Celui du jeu]
  \begin{itemize}
  \item Nombre de pas joue
  \item carte joue
  \item etc
  \end{itemize}
\end{description}

On peut représenter les statistiques de beaucoup de manière différentes, mais on priviligiera souvent une visualisation graphique, généralement plus claire.

Nous avons considéré les possibilités suivantes:\\
\begin{itemize}
\item utiliser une bibliotheque php pour génerer l'image sur une page web
\item utiliser la bibliothèque matplotlib du langage python
\item utilisation d'une bibliothèque javascript
\end{itemize}

Nous avons privilégié la portabilité et un usage relativement simple, et nous avons donc choisis de génerer une page web au format html, dans laquelle les statistiques sont mis en page par du code javascript.

Nous avons utilisé la bibliothèque Highchart qui est très puissante pour génerer des graphiques sur des pages internet. 

Pour la réalisation, nous avons définis des structures javascripts pour les graphiques, les axes, les titres... Puis nous avons définis des fonctions pour allouer et libérer la mémoire, et pour enregistrer les données. 

Notre outil prend une base de données et en extrait les données. Il génère ensuite un fichier html dont le nom est définis par l'utilisateur.
En ouvrant le fichier HTML, nous avons un premier graphique qui compare les différents bots sur leur pourcentage en moyenne des portes cachée trouvée, avec l'écart type. Par ailleurs on a un graphique qui donne les détails de chaques parties (porte génerée/porte trouvée par exemple).
