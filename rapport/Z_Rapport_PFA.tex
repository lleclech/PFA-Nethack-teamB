\documentclass[a4paper,11pt,final]{report}
% Pour une impression recto verso, utilisez plutôt ce documentclass :
%\documentclass[a4paper,twoside,11pt,final]{article}

\usepackage[francais]{babel}
\usepackage[utf8]{inputenc}
\usepackage[T1]{fontenc}
\usepackage[pdftex]{graphicx}
\usepackage{setspace}
\usepackage[colorlinks=true,linkcolor=black]{hyperref}
\usepackage[french]{varioref}
\usepackage[top=7em, bottom=7em, left=7em, right=7em]{geometry}
\usepackage{setspace}
\usepackage{wrapfig}

\newcommand{\reporttitle}{Six mois de stage chez Logica}     % Titre
\newcommand{\reportauthor}{Louis \textsc{LE CLEC'H}} % Auteur
\newcommand{\reportsubject}{Stage du second semestre} % Sujet
\newcommand{\HRule}{\rule{\linewidth}{0.5mm}}
\setlength{\parskip}{1ex} % Espace entre les paragraphes

\hypersetup{
    pdftitle={\reporttitle},%
    pdfauthor={\reportauthor},%
    pdfsubject={\reportsubject},%
   % pdfkeywords={rapport} {vos} {mots} {clés}
}

\begin{document}
\begin{onehalfspace}
  \include{Title}
  %\cleardoublepage % Dans le cas du recto verso, ajoute une page blanche si besoin
  \tableofcontents % Table des matières
  \listoffigures
 % \sloppy          % Justification moins stricte : des mots ne dépasseront pas des paragraphes
  %\cleardoublepage
  \include{Remerciements}
  
  \include{Intro}
  
  \include{Presentation_des_SSII-Logica}
  
  \include{TESTING_FACTORY}
  
  \include{Projets_et_Applications}
  
  \include{Partie_travail_personnel}
  
  \include{Conclusion}
  
  \include{Annexe}
\end{onehalfspace}
\end{document}
