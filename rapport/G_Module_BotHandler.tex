\chapter{Présentation des projets sur lesquels j'ai travaillé}

\section{SAFIR épargne (SAFIR V1)}

\subsection{Descriptif}
L'application SAFIR est utilisée dans les Bureaux de Poste et offre aux conseillers financiers un espace de travail organisé en situation de travail et processus. Elle permet, depuis le poste de travail, un accès à des services et des données dans une vision client. Dans le cadre de SAFIR V1 l'agent en bureau de poste peut initialiser l'ouverture de comptes épargne. 
SAFIR V1 comprend aussi l'application Suivi/ Reprise qui permet à un agent de reprendre un contrat dont il n'avait pas pu finir l'ouverture (à cause d'un problème dû à l'application ou d'un arrêt volontaire de sa part).

\subsection{Fonctionnalité}
Le conseiller financier en bureau de poste peut à travers ce processus saisir sur son poste de travail une demande de nouveau contrat d’épargne (type livret A, livret B, CEL,…) pour le client se trouvant en face de lui. Il pourra également, lors de ce processus, proposer au client des équipements permettant d'alimenter ou de retirer de l'argent sur le compte épargne.

Les différentes étapes du processus sont les suivantes :\\
\begin{itemize}
\item Identification du client : recherche à travers les bases de données LBP du client à partir de sa signalétique ou d’un numéro de contrat. La création du client est possible si celui est inconnu de LBP
\item Choix du produit désiré
\item Contrôle des pièces justificatives à fournir par le client
\item Saisie des données détaillées du contrat : montant de versement initial, mode de versement
\item Choix des équipements : une carte de retrait ? un versement régulier depuis un compte dépôt ?
\item Édition du contrat et récupération de la signature du contrat du client
\item Enregistrements des opérations dans les différent référentiels LBP : \begin{itemize}
												   \item SIROCCO pour le contrat
												   \item CHEOPS pour le versement initial
												   \item BAV pour les informations générales du dossier
												   \end{itemize}

\end{itemize}

\section{SAFIR dépôt (SAFIR V2)}

\subsection{Descriptif}
L'application SAFIR est utilisée dans les Bureaux de Poste et offre aux conseillers financiers un espace de travail organisé en situation de travail et processus. Elle permet, depuis le poste de travail, un accès à des services et des données dans une vision client. Dans le cadre de SAFIR dépôt l'agent en bureau de poste peut initialiser l'ouverture de compte Dépôt. Ces comptes seront ensuite définitivement ouverts ou non en Centre financier selon le profil du client.

Le conseiller financier en bureau de poste peut à travers ce processus saisir sur son poste de travail une demande de nouveau contrat dépôt (type CCP, Formule de Compte,…) pour le client se trouvant en face de lui. 
Il pourra également, lors de ce processus, proposer au client des assurances, des équipements permettant d'alimenter ou de retirer de l'argent sur le compte dépôt, de domicilier ses revenus…


\subsection{Fonctionnalité}
Les différentes étapes du processus sont les suivantes :\\
\begin{itemize}
\item Identification du/des clients : recherche à travers les bases de données LBP du client à partir de sa signalétique ou d’un numéro de contrat. La création du client est possible si celui est inconnu de LBP
\item Choix du produit désiré
\item Saisie des données détaillées du contrat : montant de versement initial, mode de versement, périodicité du relevé,…
\item Choix des équipements : une carte de paiement ? un versement régulier vers un compte épargne ?
\item Enregistrements des opérations dans les différent référentiels LBP :\begin{itemize}
												   \item SIROCCO pour le contrat
												   \item BAV pour les informations générales du dossier
												   \end{itemize}

\item Contrôle des pièces justificatives à fournir par le client
\item Édition du contrat et récupération de la signature du contrat du client
\end{itemize}

\section{SAFIR équipement (SAFIR V3)}

\subsection{Fonctionnalité}
Le conseiller financier en bureau de poste peut à travers ce processus saisir sur son poste de travail une demande de souscription d’une formule de compte pour un contrat dépôt pour le client se trouvant en face de lui. 
Il pourra également, lors de ce processus, proposer au client de modifier ses assurances, ses équipements permettant d’alimenter ou de retirer de l’argent sur le compte dépôt, sa domiciliation de ses revenus…
Les différentes étapes du processus sont les suivantes :

\begin{itemize}
\item Identification du/des clients : recherche à travers les bases de données LBP du client à partir de sa signalétique ou d’un numéro de contrat.
\item Choix de l’action désirée (souscription d’une offre et/ou modification d’équipements)
\item Choix des équipements à modifier : une carte de paiement ? un versement régulier vers un compte épargne ?
\item Enregistrements des opérations dans les différent référentiels LBP :\begin{itemize}
												   \item SIROCCO pour le contrat
												   \item BAV pour les informations générales du dossier
												   \end{itemize}
\item Contrôle des pièces justificatives à fournir par le client
\item Edition du contrat et récupération de la signature du contrat du client
\end{itemize}


\section{GCC}

L’agent en Centre Financier peut à travers ce processus ouvrir un compte de dépôt où d'épargne à un client ou bien  lui ouvrir une formule sur un compte dépôt existant. 
Fonctionnellement, le client ne se tient pas devant lui. Le COFI lui a envoyé un contrat d’ouverture de compte (imprimé papier rempli manuellement ou fait à travers SAFIR) et l’agent rentre informatiquement le contrat dans le SI
Les différentes étapes du processus sont les suivantes :



\begin{itemize}
\item Choix du produit désiré
\item Validation des données de la demande : le dossier est-il recevable ? la signature du client est-elle correcte ?
\item Identification des conditions d’octroi : en fonction de la situation et de la capacité financière du client, l’agent GCC peut dégrader ou refuser les demandes du client
\item Saisie des données détaillées du contrat : affichage en revamping des informations du contrat dans SIROCCO
\item Enregistrements des opérations dans les différents référentiels LBP 
\item Edition des courriers au client et au COFI.
\end{itemize}

\section{OASIS RC}



















